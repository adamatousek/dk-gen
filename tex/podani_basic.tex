\def\UrlZA{\url{https://is.muni.cz/el/fi/podzim2019/IB015/um/homework/zadani04.pdf}}
\def\UrlZB{\url{https://is.muni.cz/el/fi/podzim2019/IB015/um/homework/zadani08.pdf}}
\def\UrlZC{\url{https://is.muni.cz/el/fi/podzim2019/IB015/um/homework/zadani12.pdf}}
\def\UrlKA{\url{https://is.muni.cz/el/fi/podzim2019/IB015/um/homework/zadani04.hs}}
\def\UrlKB{\url{https://is.muni.cz/el/fi/podzim2019/IB015/um/homework/zadani08.hs}}
\def\UrlKC{\url{https://is.muni.cz/el/fi/podzim2019/IB015/um/homework/zadani12.hs}}

\def\st#1{\nameof{#1}, učo #1}
\def\moss#1{skóre \textsc{moss} #1~\%}
\newenvironment{zjisteni}{%
    \bigskip
    \penalty -1000
}{%
    \bigskip
    \begin{center}
        *\quad*\quad*
    \end{center}
    \bigskip
}
\long\def\Zaver{
    S~pozdravem

    \vskip 3em
    \raggedleft Adam Matoušek\\cvičící předmětu \textsc{ib}\oldstylenums{015}

    \bigskip
    \noindent V Brně dne \today
}
\makeatletter
% generate \ifhwA etc.
\define@boolkeys{opis}[hw]{ A, B, C }[true]
\edef\tmp@setkeys{\noexpand\setkeys*{opis}{\Volby}}
\tmp@setkeys % setkeys must have its argument expanded

\newcounter{PocetUloh}
\ifhwA\stepcounter{PocetUloh}\fi
\ifhwB\stepcounter{PocetUloh}\fi
\ifhwC\stepcounter{PocetUloh}\fi
\ifnum \c@PocetUloh > 1 \morehwtrue \fi
\makeatother

\def\maybeA{}
\ifmorehw
    \moreplagtrue
    \def\maybeA{\ a~\def\maybeA{}}
\fi

\def\SeznamUloh{%
    \ifhwA čtvrté\ifmorehw\maybeA\fi\fi
    \ifhwB osmé\fi\maybeA
    \ifhwC dvanácté\fi
}

\def\PodobnostiU#1{
    \par\penalty 8000
    \noindent Podobnosti v řešení\ifmoreplag ch\fi: \PodobnostMaji{#1}
    \par
    \medskip\ignorespaces%
}

\ifmorehw
    \def\KUloze#1{%
        \bigskip
        \noindent%
        \textbf{\ignorespaces%
            \if#1A  Čtvrtá\fi%
            \if#1B  Osmá\fi%
            \if#1C  Dvanáctá\fi%
            \ úloha%
        }\par\penalty 10000\medskip
        \PodobnostiU{#1}
    }
\else
    \def\KUloze{\PodobnostiU}
\fi


\begin{document}
\noindent děkan FI MU\\
prof.\ RNDr. Jiří Zlatuška, CSc.\\
Botanická 68a\\
602 00 Brno\\
\vskip6ex

\begin{center}
{\Large {\sc Podnět k projednání disciplinárního přestupku}}
\end{center}
\vskip6ex

\noindent Vážený pane děkane,

\bigskip\noindent domnívám se, že došlo k disciplinárnímu přestupku podle
článku~2, odstavce~2 Disciplinárního řádu FI MU a~podávám podnět
k~jeho projednání.

Součástí předmětu \textsc{ib}\oldstylenums{015} \textit{Neimperativní
programování} je řešení pravidelných
domácích úloh. Studenti mají povinnost tyto úkoly vypracovávat samostatně. Cituji
bod z~interaktivní osnovy předmětu\footnote{\url{https://is.muni.cz/el/fi/podzim2019/IB015/index.qwarp}}:

\begin{quote}
„Domácí úkoly vypracovávejte samostatně. Spolupráce vícero lidí je zakázána. Viz disciplinární řád FI MU.“
\end{quote}

\noindent
Obdobná informace je navíc zopakována i v zadání velkých úloh; cituji:

\begin{quote}
„Nezapomeňte, že opisování je zakázáno a bude postihováno podle disciplinárního řádu.“
\end{quote}


\noindent
Domnívám se, že toto pravidlo \ifmale porušil student\else porušila studentka\fi

\begin{center}
    \textbf{\st{\UCO}},
\end{center}

\noindent
a to sdílením či opisováním zdrojového kódu řešení \SeznamUloh\ domácí úlohy. Domnívám
se tak na základě přílišné podobnosti \ifmale jím\else jí\fi\ \ifmorehw
odevzdaných\else odevzdaného\fi\ řešení s~\ifmoreplag jinými řešeními\else jiným
řešením\fi. Přílohou tohoto podnětu
jsou příslušné zdrojové kódy. Zadání a~\ifmorehw kostry\else kostra\fi\ řešení
jsou k~nalezení ve studijních materiálech před\-mětu v~Informačním
systému\footnote{\url{https://is.muni.cz/el/fi/podzim2019/IB015/um/homework/}}.

Na podobnost kódů mne upozornil detektor plagiátů
\textsc{moss}\footnote{\url{http://theory.stanford.edu/~aiken/moss/}}. Ten
dvojicím souborům přiřazuje skóre od 0 do 100~\%, přičemž nezapočítává řádky
shodné s kostrou (kvůli tomu ani zcela shodná řešení nedosahují 100~\%).
Zdrojové kódy jsem dále podrobil ruční analyse, abych vyloučil falešná positiva.
Má zjištění jsou následující:

\begin{zjisteni}
    \Zjisteni
\end{zjisteni}

\Zaver

\bigskip
\bigskip
\raggedright Příloha: soubor \texttt{podklady-ib015-\UCO.zip}

\end{document}


